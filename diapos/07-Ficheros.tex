% Created 2022-04-26 mar 18:42
% Intended LaTeX compiler: pdflatex
\documentclass[usenames,svgnames,dvipsnames, aspectratio=169]{beamer}
\usepackage[utf8]{inputenc}
\usepackage[T1]{fontenc}
\usepackage{graphicx}
\usepackage{grffile}
\usepackage{longtable}
\usepackage{wrapfig}
\usepackage{rotating}
\usepackage[normalem]{ulem}
\usepackage{amsmath}
\usepackage{textcomp}
\usepackage{amssymb}
\usepackage{capt-of}
\usepackage{hyperref}
\usepackage{color}
\usepackage{listings}
\usepackage[spanish]{babel}
\usecolortheme{rose}
\setbeamercolor{alerted text}{fg=DarkBlue}
\setbeamerfont{alerted text}{series=\bfseries}
\setbeamerfont{block title}{series=\bfseries}
\setbeamercolor{block title}{bg=structure.fg!20!bg!50!bg}
\setbeamercolor{block body}{use=block title,bg=block title.bg}
\setbeamertemplate{navigation symbols}{\insertsectionnavigationsymbol}
\AtBeginSection[]{\begin{frame}[plain]\tableofcontents[currentsection,sectionstyle=show/shaded]\end{frame}}
\AtBeginSubsection[]{\begin{frame}[plain]\tableofcontents[currentsubsection,sectionstyle=show/shaded,subsectionstyle=show/shaded]\end{frame}}
\lstset{keywordstyle=\color{blue}, commentstyle=\color{gray!90}, basicstyle=\ttfamily\small, columns=fullflexible, breaklines=true,linewidth=\textwidth, backgroundcolor=\color{gray!23}, basewidth={0.5em,0.4em}, literate={¡}{{\textexclamdown}}1 {á}{{\'a}}1 {ñ}{{\~n}}1 {é}{{\'e}}1 {ó}{{\'o}}1 {í}{{\'i}}1 {ú}{{\'u}}1 {º}{{\textordmasculine}}1, showstringspaces=false}
\usepackage{mathpazo}
\usepackage{siunitx}
\hypersetup{colorlinks=true, linkcolor=Blue, urlcolor=Blue}
\usepackage{fancyvrb}
\DefineVerbatimEnvironment{verbatim}{Verbatim}{fontsize=\tiny, formatcom = {\color{black!70}}}
\setbeamertemplate{footline}[frame number]
\usetheme{Boadilla}
\usefonttheme{serif}
\author{Oscar Perpiñán Lamigueiro}
\date{}
\title{Tema 7: Ficheros}
\hypersetup{
 pdfauthor={Oscar Perpiñán Lamigueiro},
 pdftitle={Tema 7: Ficheros},
 pdfkeywords={},
 pdfsubject={},
 pdfcreator={Emacs 27.1 (Org mode 9.4.6)}, 
 pdflang={Spanish}}
\begin{document}

\maketitle


\section{Introducción}
\label{sec:org3123be3}

\begin{frame}[label={sec:org1063a63}]{Introducción}
\begin{itemize}
\item Hasta ahora:
\begin{itemize}
\item Introducción de datos desde el \alert{teclado}.
\item Presentación de datos en \alert{pantalla}.
\item \alert{Los datos se pierden} cuando finaliza el programa.
\end{itemize}
\item Ahora vamos a ver:
\begin{itemize}
\item \alert{Almacenamiento de datos} en ficheros que pueden ser leídos por el programa.
\item \alert{Operaciones con ficheros}: apertura, lectura y/o escritura, y cierre.
\end{itemize}
\end{itemize}
\end{frame}

\begin{frame}[label={sec:org29725ed},fragile]{Tipo \texttt{FILE}}
 En C se emplea la estructura de datos de tipo \texttt{FILE} (declarada en \texttt{stdio.h}):

\lstset{language=C,label= ,caption= ,captionpos=b,numbers=none}
\begin{lstlisting}
#include <stdio.h>

int main ()
{
  FILE *pf;

  return 0;
}
\end{lstlisting}
\end{frame}

\section{Lectura y escritura de ficheros}
\label{sec:orgab38bb6}
\begin{frame}[label={sec:org5ac2112},fragile]{Abrir un fichero: \texttt{fopen}}
 \texttt{fopen} abre un fichero para para leer y/o escribir en él.

\lstset{language=C,label= ,caption= ,captionpos=b,numbers=none}
\begin{lstlisting}
FILE *fopen (const char *nombre, const char *modo);
\end{lstlisting}

\begin{itemize}
\item \texttt{nombre}: nombre del fichero (\emph{debe respetar las normas del sistema operativo en el que se ejecute el programa}).
\item \texttt{modo}: indica cómo se va a abrir el fichero:

\begin{itemize}
\item lectura: \texttt{r}

\item escritura: \texttt{w}

\item añadir: \texttt{a}
\end{itemize}

\item Devuelve un puntero a una estructura de tipo \texttt{FILE} o un puntero nulo \texttt{NULL} si se ha producido un error.
\end{itemize}
\end{frame}

\begin{frame}[label={sec:org619bd47},fragile,plain]{Ejemplo de \texttt{fopen}}
 \lstset{language=C,label= ,caption= ,captionpos=b,numbers=none}
\begin{lstlisting}
#include <stdio.h>
int main ()
{
  FILE *pf;
  // Atención a los separadores en la ruta del fichero, 
  //y a las comillas dobles
  pf = fopen("c:/ejemplos/fichero.txt", "r");

  if (pf == NULL)
    {
      printf("Error al abrir el fichero.\n");
      return -1;
    }
  else
    {
      printf("Fichero abierto correctamente.\n");
      return 0;
    }
}
\end{lstlisting}
\end{frame}


\begin{frame}[label={sec:orgeec8079},fragile]{Cerrar un fichero: \texttt{fclose}}
 \texttt{fclose} cierra un fichero previamente abierto con \texttt{fopen}

\lstset{language=C,label= ,caption= ,captionpos=b,numbers=none}
\begin{lstlisting}
int fclose (FILE *pf);
\end{lstlisting}

\begin{itemize}
\item El puntero \texttt{pf}, de tipo \texttt{FILE}, apunta al fichero.
\item La función devuelve 0 si el fichero se cierra correctamente o \texttt{EOF}\footnote{\texttt{EOF} (\emph{End Of File}) es la marca de final de fichero. Se explica en \hyperlink{sec:org48a6f15}{la diapositiva \guillemotleft{}Marca de final de fichero\guillemotright{}}} si se ha producido un error.
\end{itemize}
\end{frame}

\begin{frame}[label={sec:orgb710bc8},fragile]{Escritura de ficheros: \texttt{fprintf}}
 \lstset{language=C,label= ,caption= ,captionpos=b,numbers=none}
\begin{lstlisting}
int fprintf(FILE *stream, const char *format, ...)
\end{lstlisting}

\begin{itemize}
\item Escribe en un fichero con el formato especificado (\alert{igual que \texttt{printf}})

\item Devuelve el número de caracteres escritos, o un valor negativo si ocurre un error.
\end{itemize}
\end{frame}


\begin{frame}[label={sec:org01437fa},fragile,plain]{Ejemplo de \texttt{fprintf}}
 \lstset{language=C,label= ,caption= ,captionpos=b,numbers=none}
\begin{lstlisting}
#include <stdio.h>

int main(){
  FILE *pf;
  int vals[3] = {1, 2, 3};
  // Abrimos fichero para escritura
  pf = fopen("datos.txt", "w");
  if (pf == NULL) {// Si el resultado es NULL mensaje de error 
    printf("Error al abrir el fichero.\n");
    return -1;
  }
  else {// Si ha funcionado, comienza escritura
    fprintf(pf, "%i, %i, %i",
	    vals[0], vals[1], vals[2]);
    fclose(pf); // Cerramos fichero
    return 0;
  }
}
\end{lstlisting}
\end{frame}


\begin{frame}[label={sec:org40d0326},fragile]{Lectura de ficheros: \texttt{fscanf}}
 \lstset{language=C,label= ,caption= ,captionpos=b,numbers=none}
\begin{lstlisting}
int fscanf(FILE *stream, const char *format, ...)
\end{lstlisting}

\begin{itemize}
\item Lee desde un fichero con el formato especificado (\alert{igual que \texttt{scanf}})

\item Devuelve el número de argumentos que han sido leídos y asignados o \texttt{EOF} si se detecta el final del fichero.
\end{itemize}
\end{frame}


\begin{frame}[label={sec:org92414cb},fragile,plain]{Ejemplo de \texttt{fscanf}}
 \lstset{language=C,label= ,caption= ,captionpos=b,numbers=none}
\begin{lstlisting}
#include <stdio.h>

int main()
{
  int i, n, vals[3];
  FILE *pf;
  // Abrimos fichero para lectura
  pf = fopen("datos.txt", "r");
  // Leemos datos separados por comas
  n = fscanf(pf, "%i, %i, %i",
	 &vals[0], &vals[1], &vals[2]);
  printf("Se han leído %i argumentos.\n", n);
  fclose(pf);
  // Mostramos en pantalla lo leído
  for (i = 0; i < 3; i++)
    printf("%i\t", vals[i]);

  return(0);
}
\end{lstlisting}
\end{frame}



\begin{frame}[label={sec:org5b983e3},fragile]{Lectura de datos con separadores}
 Si en un fichero hay datos numéricos junto con cadenas de caracteres, hay que usar \texttt{[\textasciicircum{}X]}, donde \texttt{X} es el carácter empleado como separador. 

Por ejemplo, para leer datos separados por punto y coma empleamos: \texttt{[\textasciicircum{};]}

\begin{block}{Ejemplo}
Sea un fichero con el siguiente contenido:
\begin{quote}
Jorge Rodríguez; Profesor; 35; 84.4
\end{quote}

Para leerlo:
\lstset{language=C,label= ,caption= ,captionpos=b,numbers=none}
\begin{lstlisting}
fscanf(pf, "%[^;];%[^;];%d;%f\n",
       nombre, tipo, &edad, &peso);
\end{lstlisting}
\end{block}
\end{frame}

\begin{frame}[label={sec:org48a6f15},fragile]{Marca de final de fichero \texttt{EOF}}
 \begin{itemize}
\item Cuando se crea un fichero nuevo con \texttt{fopen} se añade automáticamente al final la marca de fin de fichero \texttt{EOF} (\emph{end of file}).
\item Es una marca escrita al final de un fichero que indica que no hay más datos.
\item Cuando se realizan operaciones de lectura o escritura es necesario comprobar si se ha alcanzado esta marca.
\end{itemize}
\end{frame}


\begin{frame}[label={sec:org909d046},fragile]{Comprobación de \texttt{EOF}}
 \begin{itemize}
\item \texttt{feof} detecta el final del fichero: devuelve un valor distinto de cero después de la primera operación que intente leer después de la marca final del fichero.
\end{itemize}

\lstset{language=C,label= ,caption= ,captionpos=b,numbers=none}
\begin{lstlisting}
while (feof(pf) == 0)
{
// Operaciones de L/E
}
\end{lstlisting}

\begin{itemize}
\item \texttt{fscanf} y \texttt{fprintf} devuelven \texttt{EOF} cuando alcanzan la marca. Se puede emplear directamente este resultado (sin necesidad de \texttt{feof})
\end{itemize}
\lstset{language=C,label= ,caption= ,captionpos=b,numbers=none}
\begin{lstlisting}
while(fscanf(...) !=EOF )
{
// Sentencias
}
\end{lstlisting}
\end{frame}

\begin{frame}[label={sec:org4b8f497},fragile]{Ejemplo: número de líneas de un fichero}
 \lstset{language=C,label= ,caption= ,captionpos=b,numbers=none}
\begin{lstlisting}
#include <stdio.h>

int main()
{
  int i, nLineas = 0;
  char x; // Variable auxiliar
  FILE *pf;
  pf = fopen("lorem_ipsum.txt", "r");
  // Leemos caracter a caracter
  while (fscanf(pf, "%c", &x) != EOF)
    //Si lo leído es un salto de línea
      if (x == '\n')
	//incrementamos el contador
	++nLineas;
  printf("%i", nLineas);
  return 0;
}
\end{lstlisting}
\end{frame}

\section{Miscelánea}
\label{sec:orgbed5c56}
\begin{frame}[label={sec:orgbe6fd7d},fragile]{stdin, stdout, and stderr}
 \begin{itemize}
\item Al ejecutarse un programa de C se abren tres ficheros de forma automática (identificados por tres punteros de tipo \texttt{FILE}):

\begin{itemize}
\item \texttt{stdin}: entrada estándar del programa (habitualmente el teclado).

\item \texttt{stdout}: salida estándar del programa (habitualmente la pantalla).

\item \texttt{stderr}: fichero estándar de error.
\end{itemize}
\end{itemize}

\begin{block}{Ejemplo}
\lstset{language=C,label= ,caption= ,captionpos=b,numbers=none}
\begin{lstlisting}
#include <stdio.h>

int main(){
  printf("hello there.\n");
  fprintf(stdout, "hello there.\n");
  return 0;
}
\end{lstlisting}
\end{block}
\end{frame}


\begin{frame}[label={sec:org30e6414},fragile,plain]{Movimiento en un fichero}
 \begin{block}{\texttt{fseek}}
\lstset{language=C,label= ,caption= ,captionpos=b,numbers=none}
\begin{lstlisting}
int fseek(FILE *stream, long int offset, int whence)
\end{lstlisting}

\begin{itemize}
\item Desplaza a una posición en un fichero
\item \texttt{offset} (long): valor (en bytes) a ir desde \texttt{whence}
\item \texttt{whence}:
\begin{itemize}
\item \texttt{SEEK\_SET} Comienzo del fichero
\item \texttt{SEEK\_CUR} Posición actual
\item \texttt{SEEK\_END} Final del fichero
\end{itemize}
\end{itemize}
\end{block}

\begin{block}{\texttt{ftell}}
\lstset{language=C,label= ,caption= ,captionpos=b,numbers=none}
\begin{lstlisting}
long int ftell(FILE *stream)
\end{lstlisting}

\begin{itemize}
\item Devuelve la posición actual respecto del inicio del fichero.
\item Las unidades suelen ser \alert{bytes}.
\item Es una función de tipo \texttt{long}
\end{itemize}
\end{block}
\end{frame}

\begin{frame}[label={sec:org6cc4565},fragile]{Ejemplo: nº de bytes de un fichero}
 \lstset{language=C,label= ,caption= ,captionpos=b,numbers=none}
\begin{lstlisting}
#include <stdio.h>

int main()
{
  long int fsize; // tamaño del fichero
  FILE *pf;
  pf = fopen("datos.txt", "r");
  // Desplaza al final
  fseek(pf, 0, SEEK_END);
  //Almacena la posición
  fsize = ftell(pf);
  printf("El fichero tiene %li bytes.\n",
	 fsize);
  return 0;
}
\end{lstlisting}
\end{frame}
\end{document}