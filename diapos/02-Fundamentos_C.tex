% Created 2022-02-02 mié 21:03
% Intended LaTeX compiler: pdflatex
\documentclass[xcolor={usenames,svgnames,dvipsnames}, aspectratio=169]{beamer}
\usepackage[utf8]{inputenc}
\usepackage[T1]{fontenc}
\usepackage{graphicx}
\usepackage{grffile}
\usepackage{longtable}
\usepackage{wrapfig}
\usepackage{rotating}
\usepackage[normalem]{ulem}
\usepackage{amsmath}
\usepackage{textcomp}
\usepackage{amssymb}
\usepackage{capt-of}
\usepackage{hyperref}
\usepackage{color}
\usepackage{listings}
\usepackage[spanish]{babel}
\usecolortheme{rose}
\setbeamercolor{alerted text}{fg=DarkBlue}
\setbeamerfont{alerted text}{series=\bfseries}
\setbeamerfont{block title}{series=\bfseries}
\setbeamercolor{block title}{bg=structure.fg!20!bg!50!bg}
\setbeamercolor{block body}{use=block title,bg=block title.bg}
\setbeamertemplate{navigation symbols}{\insertsectionnavigationsymbol}
\AtBeginSection[]{\begin{frame}[plain]\tableofcontents[currentsection,sectionstyle=show/shaded]\end{frame}}
\AtBeginSubsection[]{\begin{frame}[plain]\tableofcontents[currentsubsection,sectionstyle=show/shaded,subsectionstyle=show/shaded]\end{frame}}
\lstset{keywordstyle=\color{blue}, commentstyle=\color{gray!90}, basicstyle=\ttfamily\small, columns=fullflexible, breaklines=true,linewidth=\textwidth, backgroundcolor=\color{gray!23}, basewidth={0.5em,0.4em}, literate={¡}{{\textexclamdown}}1 {á}{{\'a}}1 {ñ}{{\~n}}1 {é}{{\'e}}1 {ó}{{\'o}}1 {í}{{\'i}}1 {ú}{{\'u}}1 {º}{{\textordmasculine}}1, showstringspaces=false}
\usepackage{mathpazo}
\usepackage{siunitx}
\hypersetup{colorlinks=true, linkcolor=Blue, urlcolor=Blue}
\usepackage{fancyvrb}
\DefineVerbatimEnvironment{verbatim}{Verbatim}{fontsize=\tiny, formatcom = {\color{black!70}}}
\setbeamertemplate{footline}[frame number]
\usetheme{Boadilla}
\usefonttheme{serif}
\author{Oscar Perpiñán Lamigueiro}
\date{}
\title{Tema 2: Fundamentos de C}
\hypersetup{
 pdfauthor={Oscar Perpiñán Lamigueiro},
 pdftitle={Tema 2: Fundamentos de C},
 pdfkeywords={},
 pdfsubject={},
 pdfcreator={Emacs 27.1 (Org mode 9.4.6)}, 
 pdflang={Spanish}}
\begin{document}

\maketitle


\section{Primeros pasos}
\label{sec:org605fe6a}
\begin{frame}[label={sec:orga7b8398},fragile]{Hello World!}
 \lstset{language=C,label= ,caption= ,captionpos=b,numbers=none}
\begin{lstlisting}
#include <stdio.h>

int main()
{
  printf("Hello World!\n");

  return 0;
}
\end{lstlisting}
\end{frame}

\begin{frame}[label={sec:org255a37d},fragile]{Hello World! (2)}
 \lstset{language=C,label= ,caption= ,captionpos=b,numbers=none}
\begin{lstlisting}
#include <stdio.h>

int main()
{
  printf("Hello World!\n");
  printf("¡Hola Mundo!\n");
  printf("Bonjour le Monde!\n");
  printf("Hallo Welt!\n");

  return 0;
}
\end{lstlisting}
\end{frame}

\begin{frame}[label={sec:org19f6e3d},fragile]{Comentarios}
 \lstset{language=C,label= ,caption= ,captionpos=b,numbers=none}
\begin{lstlisting}
/** Este simple programa sirve para 
   mostrar un mensaje en pantalla */
#include <stdio.h>

// Todo programa necesita una función main.
// Su contenido está delimitado entre llaves
int main()
{
  //La función printf muestra el mensaje en pantalla
  // Atención: el mensaje debe ir entre comillas
  printf("Hello World!\n");
  // La función main devuelve un entero con return.
  return 0;
} // Aquí acaba main y por tanto el programa
\end{lstlisting}
\end{frame}

\section{Datos en C}
\label{sec:org90234b6}
\subsection{Introducción}
\label{sec:org67ddb09}
\begin{frame}[label={sec:org194296a}]{Constantes y Variables}
\begin{description}
\item[{Constantes}] datos cuyo valor no se puede modificar durante la ejecución del programa
\item[{Variables}] datos cuyo valor se puede modificar mediante el operador \emph{asignación} (=)
\end{description}
\end{frame}
\begin{frame}[label={sec:org2be2e35},fragile,plain]{Constantes y Variables}
 \lstset{language=C,label= ,caption= ,captionpos=b,numbers=none}
\begin{lstlisting}
int main()
{
  // declara una variable con el identificador v1
  int v1;
  // declara una constante simbólica
  // con el identificador c1
  const int c1 = 4;
  // declara una variable v2,
  // y le asigna el valor 2 (una constante literal)
  int v2 = 2;
  // asigna el valor de la
  // constante c1 a la variable v1
  v1 = c1;
  // ídem con v2 (cambia su valor previo)
  v2 = c1;
  // error: c1 es una constante
  c1 = 3;
  return 0;
}
\end{lstlisting}
\end{frame}
\begin{frame}[label={sec:org1f9af04}]{Nombres de constantes y variables}
\begin{itemize}
\item Primer carácter: letra o carácter de subrayado (\_) (\alert{nunca un número}).
\item Una o más letras (A-Z, a-z, \emph{ñ excluida}), dígitos (0-9) o caracteres de subrayado.
\item Tienen que ser distintos de las palabras clave.
\item Las mayúsculas y las minúsculas son diferentes para el compilador.
\item Es aconsejable que los nombres sean representativos
\end{itemize}
\end{frame}

\begin{frame}[label={sec:orgabff0db}]{Palabras clave o reservadas}
\begin{center}
\begin{tabular}{llll}
auto & double & int & struct\\
break & else & long & switch\\
case & enum & register & typedef\\
char & extern & return & union\\
const & float & short & unsigned\\
continue & for & signed & void\\
default & goto & sizeof & volatile\\
do & if & static & while\\
\end{tabular}
\end{center}
\end{frame}

\subsection{Almacenamiento de la Información}
\label{sec:org3a98476}

\begin{frame}[label={sec:org23b1f2a}]{bit}
\begin{itemize}
\item Los ordenadores utilizan el sistema de numeración binario (dos dígitos, 0 y 1) para almacenar información.

\item Un \alert{dígito binario} (0 ó 1) se denomina \emph{bit} (\emph{binary digit}).

\item Con \alert{N bits} pueden representarse \alert{2\textsuperscript{N} símbolos o 2\textsuperscript{N} números}

\begin{itemize}
\item Ejemplo: con N = 8 bits se pueden representar los números positivos desde el 0 al 255 (2\textsuperscript{8} - 1).
\end{itemize}
\end{itemize}
\end{frame}

\begin{frame}[label={sec:org0bdab1b}]{Representación de la información: binario y decimal}
\begin{block}{Ejemplo en decimal: 3452}
\begin{center}
\begin{tabular}{rrrr}
10\textsuperscript{3} & 10\textsuperscript{2} & 10\textsuperscript{1} & 10\textsuperscript{0}\\
1000 & 100 & 10 & 1\\
\hline
3 & 4 & 5 & 2\\
\end{tabular}
\end{center}

\begin{center}
3452 = 3 \(\cdot\) 1000 + 4 \(\cdot\) 100 + 5 \(\cdot\) 10 + 2 \(\cdot\) 1
\end{center}
\end{block}


\begin{block}{Ejemplo en binario: 10001111}
\begin{center}
\begin{tabular}{lrrrllll}
2\textsuperscript{7} & 2\textsuperscript{6} & 2\textsuperscript{5} & 2\textsuperscript{4} & 2\textsuperscript{3} & 2\textsuperscript{2} & 2\textsuperscript{1} & 2\textsuperscript{0}\\
\alert{128} & 64 & 32 & 16 & \alert{8} & \alert{4} & \alert{2} & \alert{1}\\
\hline
\alert{1} & 0 & 0 & 0 & \alert{1} & \alert{1} & \alert{1} & \alert{1}\\
\end{tabular}
\end{center}

\begin{center}
128 + 8 + 4 + 2 + 1 = 143
\end{center}
\end{block}
\end{frame}

\begin{frame}[label={sec:org8d90c69}]{No sólo números}
\begin{itemize}
\item Cualquier información puede representarse con un conjunto de bits.

\item ASCII (American Standard Code for Information Interchange): Estándar de 7 bits (128 caracteres), 95 caracteres imprimibles (del 32 al 126).
\end{itemize}

\begin{block}{}
\begin{center}
\begin{tabular}{rrrr}
01001000 & 01101111 & 01101100 & 01100001\\
72 & 111 & 108 & 97\\
\LARGE       H & \LARGE       o & \LARGE l & \LARGE a\\
\end{tabular}
\end{center}
\end{block}
\end{frame}

\begin{frame}[label={sec:org40cde47}]{Unidades de almacenamiento}
\begin{itemize}
\item Byte: 8 bits (2\textsuperscript{8} = 256)

\item Kilobyte (KB): 1024 bytes (2\textsuperscript{10} = 1024)

\item Megabyte (MB): 1024 KB (2\textsuperscript{20} bytes = 2\textsuperscript{10} KB)

\item Gigabyte (GB): 1024 MB (2\textsuperscript{30} bytes = 2\textsuperscript{10} MB = 2\textsuperscript{20} KB)

\item \ldots{}
\end{itemize}
\end{frame}



\begin{frame}[label={sec:org4ba0bc2},fragile]{Tipos de datos}
 \begin{description}
\item[{\texttt{int}}] números enteros
\end{description}
\begin{verbatim}
100 -41 0 12345
\end{verbatim}

\begin{description}
\item[{\texttt{float} y \texttt{double}}] números reales
\end{description}
\begin{verbatim}
3.0 101.2345 -0.0001 2.25e-3
\end{verbatim}
\begin{description}
\item[{\texttt{char}}] caracteres
\end{description}
\begin{verbatim}
's' '4' ';'
\end{verbatim}
\begin{description}
\item[{\texttt{\_Bool}}] \emph{booleanos}, 0 y 1
\end{description}
\end{frame}

\subsection{Números enteros}
\label{sec:org2f2efb6}

\begin{frame}[label={sec:org5a3b9b5},fragile]{Uso de \texttt{printf}}
 \lstset{language=C,label= ,caption= ,captionpos=b,numbers=none}
\begin{lstlisting}
#include <stdio.h>

int main()
{
  // Usamos %i para números enteros
  printf("Hoy es día %i\n", 6);
  // Y también %d
  printf("Hoy es día %d\n", 6);
  
  return 0;
}
\end{lstlisting}
\end{frame}

\begin{frame}[label={sec:orgcb7e218},fragile]{Definición con \texttt{int}}
 \lstset{language=C,label= ,caption= ,captionpos=b,numbers=none}
\begin{lstlisting}
#include <stdio.h>

int main()
{
  // int designa variable de números enteros
  int dia;
  // Asignamos un valor a la variable dia
  dia = 6;
  printf("Hoy es día %i\n", dia);

  return 0;
}
\end{lstlisting}
\end{frame}

\begin{frame}[label={sec:orgf1c9d45},fragile]{Definición y asignación}
 \lstset{language=C,label= ,caption= ,captionpos=b,numbers=none}
\begin{lstlisting}
#include <stdio.h>

int main()
{
  // Hacemos la asignación junto con la definición
  int dia = 6;
  printf("Hoy es día %i\n", dia);

  return 0;
}
\end{lstlisting}
\end{frame}

\begin{frame}[label={sec:org05c41e0},fragile,plain]{Rango de variables \texttt{int} con signo}
 \lstset{language=C,label= ,caption= ,captionpos=b,numbers=none}
\begin{lstlisting}
#include <stdio.h>
#include <limits.h>

int main() {
  printf("Un int ocupa %d bytes",
	 sizeof(int));
  printf(" y abarca desde %d hasta %d.\n",
	 INT_MIN, INT_MAX);
  printf("Un long int ocupa %d bytes",
	 sizeof(long int));
  printf(" y abarca desde %ld hasta %ld.\n",
	  LONG_MIN, LONG_MAX);
  printf("Un long long int ocupa %d bytes",
	 sizeof(long long int));
  printf(" y abarca desde %lld hasta %lld.\n",
	 LLONG_MIN, LLONG_MAX);
  return 0;
}
\end{lstlisting}
\end{frame}


\begin{frame}[label={sec:org6fb7212},fragile,plain]{Rango de variables \texttt{int} sin signo}
 \lstset{language=C,label= ,caption= ,captionpos=b,numbers=none}
\begin{lstlisting}
#include <stdio.h>
#include <limits.h>

int main() {
   printf("Un unsigned int ocupa %d bytes",
	  sizeof(unsigned int));
   printf(" y abarca desde 0 hasta %u.\n",
	  UINT_MAX);
   printf("Un unsigned long int ocupa %d bytes",
	  sizeof(unsigned long int));
   printf(" y abarca desde 0 hasta %lu.\n",
	  ULONG_MAX);
   printf("Un unsigned long long int ocupa %d bytes",
	  sizeof(unsigned long long int));
   printf(" y abarca desde 0 hasta %llu.\n",
	  ULLONG_MAX);
  return 0;
}
\end{lstlisting}
\end{frame}


\begin{frame}[label={sec:orgecc5370},fragile]{Lectura de números enteros con \texttt{scanf}}
 \lstset{language=C,label= ,caption= ,captionpos=b,numbers=none}
\begin{lstlisting}
#include <stdio.h>

int main()
{
  int num;

  printf("Escribe un número\n");

  //Atención: con scanf el nombre de la 
  //variable debe ir precedido de &
  scanf("%i", &num);

  printf("Has escrito el número %i\n", num);

  return 0;
}
\end{lstlisting}
\end{frame}

\begin{frame}[label={sec:orga79aff9},fragile]{Errores comunes con \texttt{scanf}}
 \begin{itemize}
\item Escribir dentro de la cadena de control mensajes y secuencias de escape (p.ej. \texttt{\textbackslash{}n}).
\item Olvidar poner el operador \alert{\&} delante de los argumentos cuando son variables de los tipos básicos (\texttt{int}, \texttt{float}, \texttt{double}, \texttt{char})
\item Poner un especificador de formato no compatible con el tipo del argumento.
\end{itemize}
\end{frame}

\subsection{Números reales}
\label{sec:org221760a}

\begin{frame}[label={sec:org774464f},fragile]{Uso de \texttt{printf}}
 \lstset{language=C,label= ,caption= ,captionpos=b,numbers=none}
\begin{lstlisting}
#include <stdio.h>

int main()
{
  double num = 102.30;
  // Usamos %f para números reales
  printf("Esto es un número real %f\n", num);
  // Indicamos número de decimales explicitamente
  printf("escrito con dos decimales %.2f\n", num);

  return 0;
}
\end{lstlisting}
\end{frame}

\begin{frame}[label={sec:org409f875},fragile]{Distintos formatos}
 \lstset{language=C,label= ,caption= ,captionpos=b,numbers=none}
\begin{lstlisting}
#include <stdio.h>

int main()
{
  double num = 103.56e10;
  printf("Esto es un número real %f\n", num);
  printf("... en notación científica %e\n", num);
  printf("... y de forma automática %g\n", num);
  return 0;
}
\end{lstlisting}
\end{frame}

\begin{frame}[label={sec:org35c2318},fragile]{Rango de números reales}
 \lstset{language=C,label= ,caption= ,captionpos=b,numbers=none}
\begin{lstlisting}
#include <stdio.h>
#include <float.h>

int main() {

   printf("Un float ocupa %d bytes",
	  sizeof(float));
   printf(" y abarca desde %e hasta %e.\n",
	  FLT_MIN, FLT_MAX);

   printf("Un double ocupa %d bytes",
	  sizeof(double));
   printf(" y abarca desde %e hasta %e.\n",
	  DBL_MIN, DBL_MAX);

   return 0;
}
\end{lstlisting}
\end{frame}


\begin{frame}[label={sec:org831da0f},fragile]{Lectura de números reales con \texttt{scanf}}
 \begin{block}{Identificador de Formato}
\begin{description}
\item[{\texttt{float}}] \texttt{\%f}
\end{description}

\lstset{language=C,label= ,caption= ,captionpos=b,numbers=none}
\begin{lstlisting}
#include <stdio.h>

int main()
{
  float peso, altura;

  printf("Indica tu peso (kg) y altura (m)\n");
  scanf("%f %f", &peso, &altura);

  printf("Pesas %f kg, y mides %f m.\n",
	 peso, altura);
  return 0;
}
\end{lstlisting}
\end{block}
\end{frame}


\begin{frame}[label={sec:orgcf69871},fragile]{Lectura de números reales con \texttt{scanf}}
 \begin{block}{Identificador de Formato}
\begin{description}
\item[{\texttt{double}}] \texttt{\%lf}
\end{description}

\lstset{language=C,label= ,caption= ,captionpos=b,numbers=none}
\begin{lstlisting}
#include <stdio.h>

int main()
{
  double peso, altura;

  printf("Indica tu peso (kg) y altura (m)\n");
  scanf("%lf %lf", &peso, &altura); 
  // Sin embargo, con printf siempre %f
  printf("Pesas %f kg, y mides %f m.\n",
	 peso, altura); 
  return 0;
}
\end{lstlisting}
\end{block}
\end{frame}


\subsection{Caracteres}
\label{sec:org7564a10}

\begin{frame}[label={sec:org246dc1a},fragile]{Uso de \texttt{printf}}
 \lstset{language=C,label= ,caption= ,captionpos=b,numbers=none}
\begin{lstlisting}
#include <stdio.h>

int main()
{
  // Usamos %c para caracteres
  // Atención: para delimitar caracteres usamos '
  printf("La última letra del alfabeto es la %c\n", 
	 'z');
  //Usamos %i para enteros
  printf("Su valor en la tabla ASCII es %i\n", 
	 'z');
  //Y si usamos %c para un número?
  printf("El número %i es la letra %c\n", 
	 122, 122);
  return 0;
}
\end{lstlisting}
\end{frame}

\begin{frame}[label={sec:orgf07a01b},fragile]{Definición y asignación}
 \lstset{language=C,label= ,caption= ,captionpos=b,numbers=none}
\begin{lstlisting}
#include <stdio.h>

int main()
{
  // Usamos char para asignar caracteres
  char letra = 'z';
  printf("La última letra es la %c\n", letra);
  return 0;
}
\end{lstlisting}
\end{frame}

\begin{frame}[label={sec:org905fcd9},fragile]{Asignación de números a \texttt{char}}
 \lstset{language=C,label= ,caption= ,captionpos=b,numbers=none}
\begin{lstlisting}
#include <stdio.h>

int main()
{
  // Y char con un número?
  char letra = 122;
  printf("La última letra es la %c\n", letra);
  return 0;
}
\end{lstlisting}
\end{frame}

\begin{frame}[label={sec:org1f8f864},fragile]{Lectura de caracteres con \texttt{scanf}}
 \lstset{language=C,label= ,caption= ,captionpos=b,numbers=none}
\begin{lstlisting}
#include <stdio.h>

int main()
{
  char letra;
  printf("Escribe una letra\n");
  scanf("%c", &letra);
  printf("Has escrito letra %c\n", letra);
  return 0;
}
\end{lstlisting}
\end{frame}

\section{Operadores}
\label{sec:org241de9a}
\subsection{Tipos}
\label{sec:org9360882}
\begin{frame}[label={sec:org05d078f},fragile]{Aritméticos}
 \lstset{language=C,label= ,caption= ,captionpos=b,numbers=none}
\begin{lstlisting}
x + y
x - y
x / y
x * y
x % y //módulo o resto de división de enteros
\end{lstlisting}
\end{frame}
\begin{frame}[label={sec:org3e1c0ea},fragile]{Relacionales}
 \lstset{language=C,label= ,caption= ,captionpos=b,numbers=none}
\begin{lstlisting}
x == y
x != y
x > y
x >= y
x < y
x <= y
\end{lstlisting}
\end{frame}

\begin{frame}[label={sec:org1d01f8b},fragile]{Lógicos}
 \begin{itemize}
\item AND, OR, NOT
\end{itemize}
\lstset{language=C,label= ,caption= ,captionpos=b,numbers=none}
\begin{lstlisting}
x && y //AND
x || y //OR
!x //NOT, operador unario
\end{lstlisting}

\begin{itemize}
\item Operador condicional ? (ternario)
\end{itemize}
\lstset{language=C,label= ,caption= ,captionpos=b,numbers=none}
\begin{lstlisting}
// expresión boleana ? valor si cierto : valor si falso
x > y ? "cierto" : "falso"
x == y ? "true" : "false"
\end{lstlisting}
\end{frame}

\begin{frame}[label={sec:org5d1b6a5},fragile]{Asignación}
 \lstset{language=C,label= ,caption= ,captionpos=b,numbers=none}
\begin{lstlisting}
x = y
x += y // x = x + y
x -= y // x = x - y
x *= y // x = x * y
x /= y // x = x / y
x %= y // x = x % y
\end{lstlisting}

\lstset{language=C,label= ,caption= ,captionpos=b,numbers=none}
\begin{lstlisting}
// ERROR: en el lado izquierdo no puede ir una expresión
x + y = 1
\end{lstlisting}
\end{frame}

\begin{frame}[label={sec:org169cebd},fragile]{Incrementales}
 \lstset{language=C,label= ,caption= ,captionpos=b,numbers=none}
\begin{lstlisting}
    y = ++x // x = x + 1; y = x (preincremento)
    y = x++ // y = x; x = x + 1 (postincremento)

    y = --x // x = x - 1; y = x (predecremento)
    y = x-- // y = x; x = x - 1 (postdecremento)
\end{lstlisting}
\end{frame}
\begin{frame}[label={sec:orgb658c39},fragile]{\texttt{sizeof}}
 Proporciona el tamaño de su operando en bytes.

\lstset{language=C,label= ,caption= ,captionpos=b,numbers=none}
\begin{lstlisting}
#include <stdio.h>

int main()
{
  int i1;
  float f1;
  double d1;
  char c1;

  printf("Un entero ocupa %d bytes\n", sizeof i1);
  printf("Un float ocupa %d bytes\n", sizeof f1);
  printf("Un double ocupa %d bytes\n", sizeof d1);
  printf("Un caracter ocupa %d bytes\n", sizeof c1);

  return 0;
}
\end{lstlisting}
\end{frame}

\begin{frame}[label={sec:orgd4f4283},fragile]{Bits}
 \lstset{language=C,label= ,caption= ,captionpos=b,numbers=none}
\begin{lstlisting}
x & y // Bits AND
x | y // Bits OR
x ^ y // Bits XOR
x ~ y // Bits NOT (complemento)
x << 1 // Desplazamiento de bits
x >> 1
\end{lstlisting}
\end{frame}

\subsection{Operaciones con variables}
\label{sec:org89d7b2b}
\begin{frame}[label={sec:orgcc5d7d0},fragile]{Aritméticos con enteros}
 \lstset{language=C,label= ,caption= ,captionpos=b,numbers=none}
\begin{lstlisting}
#include <stdio.h>

int main()
{
  int x, y, sum;
  x = 10;
  y = 15;
  sum = x + y;

  printf("La suma de %i con %i es %i\n",
	 x, y, sum);
  return 0;
}
\end{lstlisting}
\end{frame}


\begin{frame}[label={sec:orgb242860},fragile]{Aritméticos con caracteres}
 \lstset{language=C,label= ,caption= ,captionpos=b,numbers=none}
\begin{lstlisting}
#include <stdio.h>

int main()
{
  char letra, Letra;
  letra = 'z';
  Letra = letra - 32;

  printf("La letra %c en mayúscula es %c\n",
	 letra, Letra);
  return 0;
}
\end{lstlisting}
\end{frame}

\begin{frame}[label={sec:orgc8b69ec},fragile]{Aritméticos con números reales}
 \lstset{language=C,label= ,caption= ,captionpos=b,numbers=none}
\begin{lstlisting}
#include <stdio.h>

int main()
{
  float peso, altura, imc;

  printf("Indica tu peso (kg) y altura (m)\n");
  scanf("%f %f", &peso, &altura);

  imc = peso / (altura * altura);
  printf("Tu índice de masa corporal es %f\n", imc);

  return 0;
}
\end{lstlisting}
\end{frame}
\begin{frame}[label={sec:orgaf3d6ba},fragile,plain]{Operaciones de asignación}
 \lstset{language=C,label= ,caption= ,captionpos=b,numbers=none}
\begin{lstlisting}
#include <stdio.h>
int main() { 
     int a, b = 3; 

     a = 5; 
     printf("a = %d\n", a); 
     a *= 4; // a = a * 4
     printf("a = %d\n", a); 
     a += b; // a = a + b
     printf("a = %d\n", a); 
     a /= (b + 1); // a = a / (b+1)
     printf("a = %d\n", a); 
     a = b = 1; 
     printf("a = %d, b = %d\n", a, b);

     return 0;
} 
\end{lstlisting}
\end{frame}

\begin{frame}[label={sec:org3cdce06},fragile]{Operaciones de incremento}
 \lstset{language=C,label= ,caption= ,captionpos=b,numbers=none}
\begin{lstlisting}
#include <stdio.h>

int main() 
{ 
     int b = 2, r; 
     //Preincremento
     r = ++b; 
     printf("b = %d, r = %d\n", b, r); 
     //Postincremento
     r = b++; 
     printf("b = %d, r = %d\n", b, r); 
     
     return 0; 
}
\end{lstlisting}
\end{frame}

\begin{frame}[label={sec:orgd836648},fragile]{Operaciones de incremento}
 \lstset{language=C,label= ,caption= ,captionpos=b,numbers=none}
\begin{lstlisting}
#include <stdio.h>

int main() 
{ 
     int a = 0;
     printf("a = %d\n", ++a); 
     printf("a = %d\n", a++);
     printf("a = %d\n", a);
     printf("a = %d\n", --a); 
     printf("a = %d\n", a--);     
     printf("a = %d\n", a);     

     return 0;
}
\end{lstlisting}
\end{frame}


\begin{frame}[label={sec:org479c385},fragile,plain]{Precedencia y asociatividad}
 \lstset{language=C,label= ,caption= ,captionpos=b,numbers=none}
\begin{lstlisting}
#include <stdio.h>

int main() { 
     double a = 4, b = 7, c = 3, g = 9, result; 
 
     result = a + b * c; 
     printf( "resultado = %f\n", result); 

     result = (a + b) * c; 
     printf( "resultado = %f\n", result); 

     result = a * b / c * g; 
     printf("resultado = %f\n", result); 

     result = (a * b) / (c * g); 
     printf("resultado = %f\n", result); 

     return 0;
} 
\end{lstlisting}
\end{frame}

\begin{frame}[label={sec:orgb9a3c70},fragile]{Operaciones relacionales}
 \lstset{language=C,label= ,caption= ,captionpos=b,numbers=none}
\begin{lstlisting}
#include <stdio.h>
int main() 
{ 
     int x = 10, y = 3; 
     
     printf("x igual a y = %d\n", 
	    (x == y)); 
     printf("x distinto a y = %d\n",
	    (x != y));     
     printf("x mayor que y = %d\n",
	    (x > y)); 
     printf("x menor o igual a y = %d\n",
	    (x <= y)); 
     printf("x mayor o igual que y = %d\n",
	    (x >= y)); 
     return 0;   
}
\end{lstlisting}
\end{frame}

\begin{frame}[label={sec:org9ce2394},fragile]{Operaciones lógicas}
 \lstset{language=C,label= ,caption= ,captionpos=b,numbers=none}
\begin{lstlisting}
#include <stdio.h>
int main() 
{ 
     int a = 3, b = 2, c = 4, d = 5; 
      
     printf("resultado = %d\n",
	    (a > b) && (c < d)); 

     printf("resultado = %d\n",
	    (a < 10) || (d != 5)); 

     printf("resultado = %d\n",
	    (a != b) && (2 * d < 8)); 

     return 0;	    
} 
\end{lstlisting}
\end{frame}
\begin{frame}[label={sec:orgfc72241},fragile,plain]{Operaciones lógicas}
 \lstset{language=C,label= ,caption= ,captionpos=b,numbers=none}
\begin{lstlisting}
#include <stdio.h>
int main() 
{ 
  int x, resto;

  printf("Escribe un número entero: ");

  scanf("%d", &x);

  // Calcula el resto de dividir por 2
  resto = x % 2;

  // Si el resto es 0, x es par.
  printf("Es un número %s\n",
	 (resto == 0) ? "par" : "impar");

  return 0;
} 
\end{lstlisting}
\end{frame}

\subsection{Conversión de tipos de datos}
\label{sec:org13a3e6f}

\begin{frame}[label={sec:org9fa3ce3},fragile]{Conversión implícita}
 \begin{description}
\item[{Asignaciones}] el valor de la derecha se convierte al tipo de la variable de la izquierda (posible aviso o error).
\end{description}
\lstset{language=C,label= ,caption= ,captionpos=b,numbers=none}
\begin{lstlisting}
#include <stdio.h>

int main() {
  float f1 = 3.7, f2;
  int i1 = 2, i2;
  // Real a entero: pierde decimales
  i2 = f1;
  printf("Un real %f convertido a entero %d\n", 
	 f1, i2);
  // Entero a real: no cambia valor
  f2 = i1;
  printf("Un entero %d convertido a real %f\n",
	 i1, f2);
  return 0;
}
\end{lstlisting}
\end{frame}



\begin{frame}[label={sec:org7ea5178},fragile]{Conversión explícita}
 \begin{description}
\item[{Conversión explícita o forzada}] \texttt{(tipo) expresión}
\end{description}
\lstset{language=C,label= ,caption= ,captionpos=b,numbers=none}
\begin{lstlisting}
#include <stdio.h>

int main()
{
  float f1 = 3.7, f2;
  int i1 = 2, i2;

  f2 = (float) i1;
  printf("Un entero %d convertido a real %f\n",
	 i1, f2);

  i2 = (int) f1;
  printf("Un real %f convertido a entero %d\n",
	 f1, i2);
  return 0;
}
\end{lstlisting}
\end{frame}

\begin{frame}[label={sec:org807bb19},fragile]{Conversión en expresiones}
 \begin{description}
\item[{Expresiones}] los valores de los operandos se convierten al tipo del operando que tenga la precisión más alta.
\end{description}

\lstset{language=C,label= ,caption= ,captionpos=b,numbers=none}
\begin{lstlisting}
#include <stdio.h>

int main()
{
  double f1 = 100;
  int i1 = 150, i2 = 100;
  printf("Un entero, %d, dividido por un real, %f,",
	 i1, f1);
  printf(" produce un real, %f\n",
	 i1 / f1);
  printf("Un entero, %d, por un entero, %d: %d\n",
	 i1, i2, i1 / i2);
  return 0;
}
\end{lstlisting}
\end{frame}
\end{document}